\documentclass[a4paper, twocolumn]{article}
\usepackage[margin=1.5cm]{geometry}

\usepackage{amsmath, amsthm, amssymb, amsfonts} 
\usepackage{color}
%\usepackage{german}
\usepackage[utf8]{inputenc}
%\usepackage{multicol}  
\usepackage{multirow}
%\usepackage{dsfont} 
%\usepackage[rflt]{floatflt}  
\usepackage{graphics}
\usepackage{epsfig} 
\usepackage[pdftex]{hyperref}
\usepackage{moreverb}

\usepackage[novbox]{pdfsync}

\title{Summary: Die Kunst des klugen Handelns \\
	\large
	52 Irrwege, die sie besser anderen überlassen
	Rolf Dobelli}
\author{Stefan Götschi}

% Commands

\definecolor{lightgrey}{rgb}{0.96,0.96,0.96}
\newcommand{\code}[1]{\texttt{\color{black}{#1}}}
\newcommand{\keyword}[1]{\textbf{\color{black}{#1}}}
\newcommand{\example}[1]{\textbf{\medskip\colorbox{lightgrey}{\color{black}{Beispiel: #1}}}}

\begin{document}
\maketitle

\subsubsection*{Warum schlechte Gründe oft ausreichen}
\textbf{Begründungsrechtfertigung}\\
Menschen wollen ein "weil" haben und sind unzufrieden ohne es.

\subsubsection*{Warum Sie besser entscheiden, wenn Sie weniger entscheiden}
\textbf{Entscheidungsermüdung}\\
Entscheiden kostet Willenskraft. Viele Entscheidungen zu fällen, führt zu Entscheidungsermüdung.\\
Auch Essenszeiten beinflussen die Entscheidungsfreudigkeit - nach dem Essen ist sie höher bzw. "mutiger".

\subsubsection*{Warum Sie Hitlers Pullover nicht tragen würden}
\textbf{Berührungsdenkfehler}\\
Meschen stellen Verbindungen zwischen Personen und Dingen her, auch wenn sie nur immaterieller Art sind.

\subsubsection*{Warum es keinen durchschnittlichen Krieg gibt}
\textbf{Das Problem mit dem Durchschnitt}\\
Der Druchschnitt ist (gerade bei Geld) nicht gleich dem Median. Der Durchschnitt maskiert die Verteilung.

\subsubsection*{Wie Sie mit Boni Motivation zerstören}
\textbf{Motivationsverdrängung}
Wenn man etwas (z.B. Freundschaftsdienst, kleine Übertretung, ...) einen Preis gibt, wertet das ab.\\
Monetäre Motivation verdränkt leider die nicht-monetäre, statt sie zu unterstützen.\\
Boni (die an Ziele geknüpft sind) können von anderen Zielen ablenken.

\subsubsection*{Wenn du nichts zu sagen hast, sage nichts}
\textbf{Plappertendenz}\\
Geplapper (ob mit schönen Worten oder oberflächlich) maskiert Nichtwissen. Einfache Erklärungen sind schwer.

\subsubsection*{Wie Sie als Manager bessere Zahlen ausweisen, ohne etwas dafür zu tun}
\textbf{Will-Rogers-Phänomen}\\
Wenn man zwei unterschiedlich gute Sets and Dingen hat, kann man (häufig) eines der Schlechteren aus dem besseren Set ins schlechtere Set schieben und der Durchschnitt beider Sets wird besser.

\subsubsection*{Hast du einen Feind, gib ihm Information}
\textbf{Information Bias}\\
Mehr Information führt nicht immer zu besseren Entscheidungen. Mehr Information kann (in konstruierten Beispielen) zu schlechteren Entscheidungen führen.\\
\textit{Imho ist dieses Kapitel falsch: Es suggeriert, dass es kein "zu wenig" Informationen gibt. Vielmehr sollte es darum gehen die richtige Menge an Informationen zu finden.}

\subsubsection*{Warum Sie im Vollmond ein Gesicht sehen}
\textbf{Clustering Illusion}\\
Wir suchen überall nach Mustern/Strukturen, da wir dann etwas verstehen können. Dem Zufall trauen wir nicht.

\subsubsection*{Warum wir lieben, wofür wir leiden mussten}
\textbf{Aufwandsbegründung}\\
Es passiert schnell, dass eine kognitive Dissonanz zwischen unserer gefühlten Leistung und der tatsächlichen Leistung besteht.\\
Man kann das z.B. an Initiationsriten sehen.

\subsubsection*{Warum kleine Filialen aus der Reihe Tanzen}
\textbf{Das Gesetz der kleinen Zahl}\\
Statistik: Absolute Abweichungen haben auf kleineren Zahlen eine viel grössere Auswirkung als auf grössere.\\
Z.B. machen Diebstahlsraten bei kleinen Geschäften mehr aus, als bei grösseren.

\subsubsection*{Gehen sie mit Ihren Erwartungen vorsichtig um}
\textbf{Erwartungen}\\ 
Wir passen unser Verhalten unseren Erwartungen an. Höhere Erwartugen an sich selbst können zu höhrer Leistung führen. An alles, das man nicht beinflussen kann, sollte man keine Erwartungen haben.

\subsubsection*{Glauben Sie nicht jeden Mist, der Ihnen spontan einfällt}
\textbf{Einfache Logik}\\
Rationales Abwägen benötigt Willenskraft - die sollte man auch bei einfacher Logik aufbringen.

\subsubsection*{Wie Sie einen Scharlatan entlarven}
\textbf{Forer-Effekt}\\
Beschreibungen, die auf viele Menschen zutreffen oder unspezifisch/offen sind, empfinden wir trotzdem als sehr zutreffend auf uns.\\
Positive Beschreibungen akzeptieren wir, auch wenn sie falsch sind. Zusätzlich kommt unser Confirmation Bias dazu, der das nicht passende unterbewust rausfiltert.

\subsubsection*{Warum Freiwilligenarbeit etwas für Stars ist}
\textbf{Voluteer's Folly}\\
Freiwilligenarbeit ist wirtschaftlicher Schwachsinn. Meistens wäre es effizienter, Überstunden zu machen und den Lohn zu spenden.\\
Dies wird der Freiwilligenarbeit aber nicht gerecht. Es geht auch um soziale Kontakte, Spass und neue Erfahrungen. Entsprechend ist es nicht reiner Altruismus.
\textit{Ich finde das Kapitel doof. Es geht um unsere Freizeit. Finanziell könnte es effizienter sein - Freiwilligenarbeit gibt uns aber Ausgleich zum Job.}

\subsubsection*{Warum Sie eine Marionette ihrer Gefühle sind}
\textbf{Affektheuristik}\\
Pro/Con Listen wären schön. Aber Menschen funktionieren über Heuristiken - Gefühle.

\subsubsection*{Warum Sie Ihr eigener Ketzer sein sollten}
\textbf{Introspection Illusion}\\
Wenn es um unsere Überzeugungen geht, sind wir nicht objektiv. Wenn dies auf die Probe gestellt wird, gibt es drei Reaktionen:
\begin{enumerate}
	\item Ignoranz-Annahme: Dem anderen fehlt die nötige Information.
	\item Idiotie-Annahme: Der andere ist nicht fähig die nötigen Schlüsse zu ziehen.
	\item Bosheits-Annahme: Der andere ist absichtlich anderer Meinung.
\end{enumerate}

\subsubsection*{Warum Sie Ihre Schiffe verbrennen lassen sollten}
\textbf{Die Unfähigkeit, Türen zu schliessen}\\
Wenn man sich alle Optionen offen hält, ist man nur halbherzig dabei. Jede Ablenkung braucht mentale Energie. Daher ist es besser Optionen zu entfernen, als sie offen zu lassen.
\textit{Ja und nein: Meist ist es dumm, sich Auswege absichtlich zu nehmen.}

\subsubsection*{Warum wir Gutes gegen Neues eintauschen}
\textbf{Neomanie}\\
Bewährtes wird sich auch weiter bewähren. Wir überschätzen die Rolle von Neuem um der Neuheit Willen.

\subsubsection*{Warum Propaganda funktioniert}
\textbf{Schläfereffekt}\\
Propaganda funktionert kurzfristig nicht, entfaltet aber langfristig ihre Wirkung. Die aktuelle Vermutung ist, dass der Zerfall der Information langsamer ist als der Zerfall des Wissens über die Quelle.\\
Lösungsversuch: Keine unverlangten Ratschläge annehmen. \textit{Muss man mindestens relativieren.} Werbequellen aus dem Weg gehen. \textit{duh?!} Versuchen, sich an die Quelle zu erinnern. \textit{duh?!}

\subsubsection*{Warum Sie oft blind für das Beste sind}
\textbf{Alternativenblindheit}\\
Story: Mach einen MBA für 100k, dann verdienst du 400k mehr $\rightarrow$ plus 300k.
\begin{enumerate}
	\item Sampling-Bias: Die 400k mehr kommen auch vom Typ Mensch - nicht nur vom MBA.
	\item Opportunitätskosten: Man kann nicht arbeiten während dem MBA $\rightarrow$ versteckte Kosten.
	\item 30 Jahre Vorhersage über Ausbildungen sind dumm. Wer weiss, ob sie dann noch "gelten".
	\item Die Entscheidung ist nicht MBA oder kein MBA. Es gibt auch andere Alternativen.
\end{enumerate}

\subsubsection*{Warum wir schlecht über die Aufsteiger reden}
\textbf{Social Comparison Bias}\\
Man sollte nicht Angst haben vor Leuten, die besser sind als man selbst. Man sollte sie einstellen, da sie dann der Firma helfen.
Dunning-Kruger-Effekt: Inkompetente Menschen können das Ausmass ihrer Inkompetenz nicht erkennen.

\subsubsection*{Warum der erste Eindruck täuscht}
\textbf{Primär- und Rezenzeffekt}\\
Wir bewerten die ersten Eindrücke stärker als spätere, wenn direkt danach abgefragt. Wenn aber etwas Zeit dazwischen liegt, sind die letzten Eindrücke die stärksten.\\
Dies kann man sich in Meetings entsprechend zunutze machen.

\subsubsection*{Warum wir kein Gefühl für das Nichtwissen haben}
\textbf{Aderlasseffekt}\\
Wir geben eine falsche Theorie erst auf, wenn eine Bessere kommt. Wir erfinden lieber Theorien als zuzugeben etwas nicht zu wissen.

\subsubsection*{Warum selbst gemacht besser schmeckt}
\textbf{Not-Invented-Here-Syndrom}\\
Man findet eigene Ideen besser. Wenn möglich sollte man Teams teilen um Ideen der einen Hälfte durch die andere Hälfte beurteilen zu lassen.

\subsubsection*{Wie Sie das Undenkbare nutzen können}
\textbf{Der Schwarze Schwan}\\
Es gibt Bekannte, bekannte Unbekannte und unbekannte Unbekannte. Pläne werden meist durch unbekannte Unbekannte umgestossen.\\
\textit{Wir sollen also alle Angsthasen sein, weil jederzeit etwas passiern könnte?!}

\subsubsection*{Warum ihr Wissen nicht transportierbar ist}
\textbf{Domain Dependence}\\
Erkenntnisse und Wissen von einem Bereich sind schwer auf einen Anderen anwendbar. E.g. kreative Marketing Leute müssen nicht kreative CEOs sein.
\textit{Oh really?!}

\subsubsection*{Warum Sie denken, die anderen würden so denken wie Sie}
\textbf{Falscher-Konsens-Effekt}\\
Die Evolution interessiert sich nicht für die Wahrheit sondern nur dafür, was es braucht zu überleben. Selbstüberzeugung hingegen schon.\\
Man sollte daher nicht von sich auf andere schliessen.

\subsubsection*{Warum Sie schon immer Recht hatten}
\textbf{Geschichtsfälschung}\\
Die Geschichte wird dauernd umgeschrieben und wir passen unsere Ansichten dauernd an. Erinnerungen sind unzuverlässig.

\subsubsection*{Warum Sie sich mit Ihrem Fussballteam identifizieren}
\textbf{In-Group/Out-Group Bias}\\
Gruppen werden aus trivialen Kriterien gebildet. Von Innerhalb scheinen Leute ausserhalb homogener. Man unterstützt die Ansichten der Leute in der Gruppe eher (Betriebsblindheit).

\subsubsection*{Warum wir nicht gerne ins Blaue hinaussegeln}
\textbf{Ambiguitätsintoleranz} \\
Wir mögen bekannte Wahrscheinlichkeiten mehr als unbekannte. Wir sollten uns vor vor Wahrscheinlichkeiten über Unbestimmtheiten hüten.
\textit{Es gibt inzwischen über fast alles historische Statistiken - daraus kann man schon Wahrscheinlichkeiten ableiten.}

\subsubsection*{Warum uns der Status Quo heilig ist}
\textbf{Default-Effekt}\\
Die Standardeinstellung ist verlockend, da bequem. Wenn es einen Standard gibt, weden die Meisten diesen wählen.

\subsubsection*{Warum die "letzte Chance" Ihren Kopf verdreht}
\textbf{Die Angst vor Reue}\\
Wir haben Angst davor, dass wir eine Entscheidung von heute in der Zukunft bereuen. Die letzte Chance verstreichen zu lassen ist eben eine Nicht-Entscheidung.\\
Wenn wir Aussenstehende sind, haben wir Mitleid, wenn eine Entscheidung von heute in der Zukunft schlecht herauskam.

\subsubsection*{Warum auffällig nicht gleich wichtig ist}
\textbf{Salienz-Effekt}\\
Das auffälligste Merkmal ist häufig nicht das Wichtigste.

\subsubsection*{Warum Probieren über Studieren geht}
\textbf{Die andere Seite des Wissens}\\
Geschriebenes Wissen hat drei Probleme:
\begin{enumerate}
	\item Sie gaukeln falsche Sicherheit vor, statt das Risiko aufzuzeigen.
	\item Wer Bücher schreibt, denkt anders, als wer Bücher liest. Bücher sind daher nicht repräsentativ.
	\item Man kann Fähigkeiten mit Worten vortäuschen (und umgekehrt).
\end{enumerate}

\subsubsection*{Warum Geld nicht nackt ist}
\textbf{House Money Effect}\\
Wir behandeln nicht alles Geld gleich - je nach dem, wie wir dazu gekommen sind. Wir sollten aber.\\
Man geht mehr Risiko ein, wenn man 30 gewinnt und einen Münzwurf nochmals 9 bringen oder verlieren kann, als wenn man zwischen einem sicheren Gewinn von 30 oder einen Münzwurf für 21 bzw. 39 wählen kann.

\subsubsection*{Warum Neujahrsvorsätze nicht funktionieren}
\textbf{Prokrastination}\\
Arbeiten, bei denen eine zeitliche Kluft zwischen Aufwand und Ertrag liegt, erforden mentale Kraft. Deshalb prokrastinieren wir sie.

\subsubsection*{Warum Sie Ihr eigenes Königreich brauchen}
\textbf{Neid}\\
Neid macht keinen Spass und ist darum "dumm". Als Referenz benutzt man immer die aktuelle Umgebung.

\subsubsection*{Warum Sie lieber Romane lesen als Statistiken}
\textbf{Presonifikation}\\
Wir reagieren kalt auf Statistiken und warm auf Menschen. Es reicht schon die Menschen nicht mehr zu sehen, damit wir diese Wärme verlieren.

\subsubsection*{Warum Krisen selten Chancen sind}
\textbf{Was-mich-nicht-umbringt-Trugschluss}\\
Eine Krise schwächt immer. Zu sagen, sie hätte einem gestärkt ist ein Trugschluss.

\subsubsection*{Warum Sie gelegentlich am Brennpunkt vorbeischauen sollten}
\textbf{Aufmerksamkeitsillusion}\\
Wir haben nur eine begrenzte Menge an Aufmerksamkeit. Ist sie schon benutzt, übersehen wir selbst offensichtliche Dinge.

\subsubsection*{Warum heisse Luft überzeugt}
\textbf{Strategische Falschangaben}\\
Je mehr auf dem Spiel steht, desto mehr wird übertrieben. Speziell bei Megaprojekten, bei denen niemand die Verantwortung trägt, viele Unternehmen einbindet und deren Abschluss erst in ein paar Jahren ist.\\
Das Problem ist, dass das auf dem Papier beste Projekt ausgewählt wird statt dem eigentlich Besten.\\
Glaube nicht, was auf dem Papier steht, sondern der Geschichte der Person/Firma und schreibe scharfe Konventionalstrafen in den Vertrag.

\subsubsection*{Wann Sie Ihren Kopf ausschalten sollten}
\textbf{Zu viel denken}\\
Nachdenken kann den Kopf von den Gefühlen abschneiden und zu seltsamen Resultaten führen. Dies ist meist besser ausser bei eingeübten Fähigkeiten oder alltäglichem Wissen.

\subsubsection*{Warum Sie sich zu viel vornehmen}
\textbf{Planungsirrtum}\\
Wir ignorieren systematisch Unplanbares wenn wir unsere Tages-/Projektplanung machen. Also Wunschdenken. Premortems können Probleme finden.

\subsubsection*{Der Mann mit dem Hammer betrachtet alles als Nagel}
\textbf{Déformation professionnelle}\\
Man findet vor allem Probleme im eigenen Metier. Man sollte auch ein paar Denkmuster ausserhalb seines Fachbereichs kennen.

\subsubsection*{Warum Pläne beruhigen}
\textbf{Zeigarnik-Effekt}\\
Offene Aufgaben nagen an uns - aber nur so lange, bis wir einen Plan dafür haben. Weitere Lektüre: David Allen.

\subsubsection*{Das Boot, in dem du sitzt, zählt mehr als die Kraft, mit der du ruderst}
\textbf{Fähigkeitsillusion}\\
Beim Unternehmertum ist Glück wichtiger als die Fähigkeiten (es geht aber nicht ohne Fähigkeiten). Für andere Dinge (z.B. Handwerker) sind Fähigkeiten wichtiger. Und in wieder anderen Bereichen regiert der Zufall.

\subsubsection*{Warum Checklisten blind machen}
\textbf{Feature-Positive Effect}\\
Präsenz ist offensichtlicher als Absenz. Fehlende Sachen auf Checklisten sind schwer zu sehen.

\subsubsection*{Warum die Zielscheibe um den Pfeil herumgemalt wird}
\textbf{Rosinenpicken}\\
Überall werden die guten Dinge gezeigt und die Schlechten versteckt. Man sollte nach den Fehlschlägen fragen.

\subsubsection*{Die steinzeitliche Jagt auf Sündenböcke}
\textbf{Die Falle des einen Grundes | Fallacy of the single cause}\\
Nichts hat nur einen Grund. Bei einem Flop muss man die beeinflussbaren Gründe finden und variieren. Gleich kann man nicht einen Sündenbock verantwortlich machen.

\subsubsection*{Warum Raser scheinbar sicherer fahren}
\textbf{Intention-To-Treat-Fehler}\\
In Studien/Statistiken können versteckte Faktoren den Ausgang beeinflussen. Z.B. zählen Unfallfahrer nicht als Raser, wenn die Zeit von A nach B gilt - sie haben ja schliesslich lange. Patienten denen es schlecht geht, landen in einer anderen Gruppe, z.B. wegen unregelmässiger Einnahme.

\subsubsection*{Warum Sie keine News lesen sollten}
\textbf{News-Illusion}\\
News brauchen viel Zeit zur Konsumation, bringen aber nur einen minimalen Vorteil. Kosten-Nutzen stehen in keinem Verhältnis.\\
\textit{Wo fangen wir an?! Ausser der Zeitersparnis ist so ziemlich alles andere kurzsichtig oder falsch.\\
Man bekommt alles von den Freunden mit: Aber auch nur, wenn die noch News lesen - sonst gibts nur Ignoranz.\\
Dazu übergibt man die Kontrolle darüber, was für News man bekommt komplett seinem Bekanntenkreis. Ist man in einem Schwurbler-Sumpf, kriegt man nur noch Schwrubler Nachrichten.\\
Selbst wenn man eine vernünftige Auswahl an Themen von den Peers bekäme, ist die Auswahl der Fakten dazu immernoch eingeschränkt auf deren Wissen.\\
Der Autor behauptet, dass Personen auf Nachfrage typischerweise 2 'wichtige' Artikel nennen können - von 10000 Artikeln, die man im letzten Jahr gelesen habe. Das ist aber eine falsche Statistik. Man würde wohl die gleichen 2 Artikel bekommen, wenn man aus der Auswahl des letzten Monats fragen würde oder 1 Artikel, für die letzte Woche. Eine deutlich bessere Bilanz.\\
Immerhin relativiert der Autor in den letzten zwei Sätzen quasi alles, was er vorher gesagt hat: Lesen Sie stattdessen lange Hintergrundartikel und Bücher. Dies ist aber geistig anstrengender - zumindest für non-fiction. Entsprechend auch kein 1:1 Ersatz.\\
Für mich klingt dieser letzte Punkt als sei er ein Test für alle, die das Buch gelesen haben, ob sie auch selber denken können.}

\end{document}
