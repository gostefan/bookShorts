\documentclass[a4paper, twocolumn]{article}

\usepackage{amsmath, amsthm, amssymb, amsfonts} 
\usepackage{color}
\usepackage[utf8]{inputenc}
\usepackage{multirow}
\usepackage{graphics}
\usepackage[pdftex]{hyperref}
\usepackage{moreverb}

\usepackage[novbox]{pdfsync}

\usepackage[framemethod=tikz]{mdframed}
\mdfdefinestyle{mystyle}{%
  rightline=true,
  innerleftmargin=10,
  innerrightmargin=10,
  outerlinewidth=3pt,
  topline=false,
  rightline=true,
  bottomline=false,
  skipabove=\topsep,
  skipbelow=\topsep
}

\title{Summary: Good Economics for Hard Times \\
	\large
	Better Answers to Our Biggest Problems\\
	Abhijit V. Banerjee \& Esther Duflo}
\author{Stefan Götschi}

% Commands

\let\oldsubsection\subsection

\definecolor{lightgrey}{rgb}{0.96,0.96,0.96}
\newcommand{\code}[1]{\texttt{\color{black}{#1}}}
\newcommand{\keyword}[1]{\textbf{\color{black}{#1}}}
\newcommand{\example}[1]{\textbf{\medskip\colorbox{lightgrey}{\color{black}{Beispiel: #1}}}}

\begin{document}
\maketitle

\section{MEGA: Make Economics Great Again}
\begin{mdframed}[style=mystyle,frametitle=Core Message]
Economists should not to just share their conclusions but also the thinking path to it.
\end{mdframed}

General standpoints/ideologies of people (e.g. gender roles) are better predictors of their policiy views than statistical data (e.g. income, home town, demographic group) \cite{hidden-tribes}. The most central topics are Immigration, trade, taxes, and the role of government.

The broad population often disagrees with Economists \cite{eco-vs-american}. Even informing people about prominent economist's views doesn't change their opinion.

Media often calls on the loudest economists. Sadly these are typically not the best but still drive public discourse.

Economy is still a combination of intuition based in science, some guesswork aided by experience, and a bunch of pure trial and error. That means economists often get things wrong.

All of these reasons lead to people not trusting economists. The only approach that helps here is if Economists explained their conclusions - The path there is at least equally important.

In this book we shouldn't forget humans want more than money and consumer goods. They want a good life. This includes dignity, family, respect, lightness, pleasure, etc.

\section{From the Mouth of the Shark}
People have a wrong perception of migrants. They think there are more of them than there actually are, that immigrants are less educated, poorer more likely to be unemployed and welfare dependent than they actually are.

The voting behavior doesn't change with fact checking politician's wrong claims.

The logic that the labour market behaves like the free market is nice and easy. More labor supply means lower wages. But this fact is wrong as we will see.

There is no evidence that even large influx of low-skilled immigrants hurt the local population.

\subsection{Leaving Home}
Immigrants don't come from the poorest countries in the world. People leave because they find it intolerable because everyday normality collapses.

In years with bad harvests less people migrate because they can't afford it \cite{push-and-pull}.
Bad years mean less people move because they can't afford it.

Most people don't move just because they can earn more. (There are exceptions as always.) Neither do Indians move from the countryside to Delhi (which would double their income) nor did the Greeks emigrate massively while unemployment was at 27 percent in 2013-2014 \cite{greek-emigration}.

\subsection{The Migration Lottery}
With migration we focus on the wages of migrants and not on the reasons why they move. The \keyword{identification problem} describes that you cannot compare the migrants with the non-migrants from the same location. The migrants may take more risks, have more stamina, or other skills that would have made them successful even without migrating.

To claim the difference in wages is caused by the difference in location and nothing else a connection between the cause and the effect have to be established without tainting factors.

Visa lotteries allow for such comparison. In New Zealand it was found that migrants from Tonga tripled their wage \cite{new-zealand-lottery}. Indian software professionals make six times more money in the United States \cite{usa-lottery}.

\subsection{Lava Bombs}
A volcanic eruption in Iceland destroyed some buildings in a fishing village while other survived. It was possible to assume that nothing distinguished the people affected.

Young people moving (typically with their parents) earned about \$3'000 more than those staying. This is mostly because they likely attended college and didn't become fishermen \cite{iceland-volcano}.

After WWII in Finland a large number of Finns had to be evacuated as land had to be conceded to Russia. Twenty-five years after the move the displaced people tended to be richer. This was mostly because they were more mobile, in more urban regions and formally employed \cite{finland-ww2}.

\subsection{Do They Know?}
The question now is why aren't poorer people moving? Aren't they aware of the opportunity?

In Bangladesh a study \cite{monga-migration} offered rural people the transportation cost to the city during "hunger season". 22 percent took it and most found employment. They brought home far more than the cost of transportation or what they earned at home. But even so only half went back to the city the next year.

So despite knowing the facts people tend to \emph{not} move.

\subsection{Lift all the Boats?}
Is migration at the expense of the natives?

Plotting the wages of natives against the share of migrants you see an upward curving slope with more migrants. But this might be because migrants move to the best opportunities.

A study \cite{mariel-boatlift} around the Mariel boatlift looked at what happened when the Miami labour force increased by 7\% (mostly low skilled workers) due to a political decision in Cuba. Comparing with other US the wages of the natives were not affected. The study remains contended - most intensely by George Borjas.

There are a couple of similar studies and they all found "very little adverse impact on the local population".

\begin{mdframed}[style=mystyle,frametitle=US National Academy of Sciences]
Empirical research in recent decades suggests that findings remain by and large consistent with those in The New Americans National Research Council (1997) in that, when measured over a period of more than 10 years, the impact of immigration on the wages of natives overall is very small. \cite{immigration-consequences}
\end{mdframed}

\subsection{What's so Special about Immigrants?}
The supply-demand graph doesn't completely translate to migration because migrants also require work done. (E.g. they need a haircut, go to a restaurant, etc.)

If the "immigrating" workers are commuting back to their country to spend their money, this additional work requirement get's lost as an episode between Germany and Czech Republic showed.

Having more low-skilled workers slows down mechanization because these workers are cheaper. This was seen when California threw out Mexican field workers and mechanization was completely adopted within 3 years \cite{mechanization}.

Native low-skilled workers can change their occupations. They speak the local language while migrants don't. So they are higher skilled than the immigrants and shift to non-manual/more communication intensive jobs \cite{native-low-skilled}.

immigrants are willing to perform tasks natives are reluctant to carry out. This drives prices of such jobs as lawn mowing, babysitting etc. down and freeing natives' time to take on other jobs \cite{low-skilled-wages} - especially highly skilled women \cite{high-skilled-women}.

Because its hard to overcome the system of immigration control many immigrants bring exceptional talents that will help them become job creators. 43\% of the 500 largest companies were (co-)founded by immigrants \cite{immigrant-founders}.

\subsection{Workers and Watermelons}
Work is not an ordinary commodity. A work relationship lasts longer than the typical commodity (e.g. Watermelon). The quality of a worker is harder to judge than that of a commodity.

Relatively few companies hire those who just walk in and ask for a job. (Even if they accepted a lower wage.)

Because companies typically ask for references and referrals, established workers are quite secure from competition from newcomers.

Firms should pay more than the minimum the worker accepts because firing them wouldn't have the punitive effect (and also because of "They pretend to pay us, we pretend to work"). Economists call this the \keyword{efficiency wage}. And because this wage is not much less than what they pay the established workers the incentive to higher a migrant is low.

Migrants typically only force new native workers out of jobs - not established workers.

\subsection{The Skilled Set}
Highly skilled migrants (e.g. Doctors, Nurses) can lower the wages of the natives significantly. But at the same time services (Doctor's Appointment) get cheaper. This worsens the labor market prospects of the native workers with similar skills.

\subsection{What Caravan?}
Summary so far that people typically don't move unless there is a disaster pushing them.

\subsection{Without Connections}
People generally move to where they might know people.

If an employer is used to employees coming with recommendations he wouldn't hire one without. If all employers work like this an employee without recommendation is ruined because only the worst employees search jobs on the open market. This is called \keyword{adverse selection}.

\subsection{The Comforts of Home}
Lots of people in developing countries live in slums even though they could afford a bit more. But there are often several missing rungs in the quality ladder of housing. This is because only small parts of cities have decent-quality infrastructure and the land prices there are astronomical. India further limits building heights very harshly which makes cities larger and commutes longer.

The low-income migrant can then chose: Crowd into a slum, commute many hours a day from his home village, or sleep under a bridge or similar.

This combined with the prospect of working jobs nobody wants discourages everyone but those who can see past the initial obstacles and pain onto climbing the ladder for better times.

The poor often need the safety net of a community. If they move to the city they lose that. You only leave if you can afford the risk of leaving.

\subsection{Family Ties}
Families have many advantages to mitigate risk. But they also have disadvantages.

Enforcing your share in a family business might be financially bad for the whole family but better for you. (\$500 for 4 brothers - if you work you'll get paid \$100 - which is less than your share.)

Some parents under-invest in the education of their children to keep them close.

\subsection{Sleepless in Kathmandu}
Migrants typically overestimate the risks of migration as well as the prospects. \cite{nepal-misinfo} shows this for Nepal.

In general the confusion and misinformation can be stronger for either aspect and either boost or dampen migration.

\subsection{Risk versus Uncertainty}
Migrants often cannot quantify decisions they need to make which makes them hesitant.

\subsection{Through a Glass Darkly}
The status quo is the natural benchmark. If you migrate you could fail and be worse off than if you didn't. People hate this kind of personal mistakes. This concept is called \keyword{loss aversion}. (This is the reason why most people are over-insured.)

A would be migrant who stays home can always claim that he would have succeeded had he gone.

\subsection{After Tocqueville}
Until about the 1990s people moved around the US a lot - there was a lot of \keyword{restlessness}. But since then the labour market has become more segregated by skill level and lead to a sense of dislocation, with some regions left behind.

The reason for this segregation is that the wage gain from being in a booming city is lower for min6-class workers than for high-skilled workers \cite{wage-gap}. (E.g. a barista makes \$12 in Boston and \$9 in Boise but the living costs are much higher in Boston.) Therefore low to middle skill level workers have more money to use if they live outside the city.

If the economy turns and a worker looks for a new job he typically likes to look where he is currently because he has support from Family (e.g. grand-parents looking after kids). Sometimes trauma is added because people are fired from the only job they did in their life and need to re-start their life.

\subsection{The Comeback Cities Tour}
There were initiatives to bring work and workers back to the "heartland" of the US. But it's a chicken and egg problem: high educated workers will not come unless amenities exist but amenities can only thrive if enough workers live there. In the beginning for every open position a worker would have to move the city - they need convincing that it's worth the risk.

\subsection{Eisenhower and Stalin}
Mobility (internal and international) helps to even out standards of living. But forced mobility hurts the most vulnerable people. The fears of locals are typically about lower wages and cultural assimilation. When locals' fears get too large it typically yields stricter migration laws - until the foreign people are integrated and part of the locals.

We should help people willing to move instead of putting up obstacles. The best way is to ease integration.

\section{The Pains from Trade}

Economics agree that imposing tariffs is bad for the economy \cite{steel-tariffs}. But the public is not convinced \cite{people-like-economics}. They feel the USA is too open to trade.

The general proposition by economists is: Everyone is better off if everyone does what they do best.

\subsection{Stan Ulam's Challenge}\label{stolper-samuelson}
A country has a \keyword{Comparative Advantage} in industries they are \emph{relatively} good at producing. Trade doesn't work if only one party produces goods. They need something in return. Therefore every country will produce the goods it can make the best even if some other country could make it better. That means it would be worse at producing any other good. There is no way one participant will win in all markets. \cite{papers-samuelson} \& \cite{political-economy}

The next step is that countries abundant in labourers will have a comparative advantage in producing labour intensive goods. And vice versa capital abundant countries will produce capital intensive goods.\cite{protection-wages}

\subsection{Beauty is Truth, Truth Beauty}
When India opened to trade (in 1991) it had one bad year and then went back to GPD growth numbers of 6\% and finally to 7.5\% in the 2000s. It's not obvious whether it's a case for or against opening to trade.

\subsection{Whereof one cannot speak, Thereof one must be silent}
It is hard to figure out what was the reason for GPD growth. Was reducing tariffs pushing growths? Or was growth pushing away tariffs? How much is due to social changes?

Comparison with different countries is hard because they all have a huge set of changes in which they differ. Looking at all low to middle income countries the Stolper Samuelson logic from section \ref{stolper-samuelson} doesn't hold tight. Low skilled workers should be off better but typically are off worse. Much worse, inequality can be linked time-wise to trade liberalization.

Corellation is not causation. China opened up at about the same time. Possibly China was a competition for low skill workers.

\subsection{The Fact that could not be}

\subsection{The sticky Economy}

\subsection{Protection for Whom}

\subsection{What's in a Name}

\subsection{The World of Names}

\subsection{The company you keep}

\onecolumn{
\bibliography{References}{}
\bibliographystyle{alpha}
}
\end{document}
