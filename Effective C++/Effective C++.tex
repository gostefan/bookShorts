\documentclass[a4paper, twocolumn]{article}

\usepackage{amsmath, amsthm, amssymb, amsfonts} 
\usepackage{color}
\usepackage[utf8]{inputenc}
\usepackage{multirow}
\usepackage{graphics}
\usepackage[pdftex]{hyperref}
\usepackage{moreverb}

\usepackage[novbox]{pdfsync}

\title{Summary: Effective C++ \\
	\large
	55 Specific Ways to Improve Your Programs and Designs
	Scott Meyers}
\author{Stefan Götschi}

% Commands

\let\oldsubsection\subsection

\newcounter{mysubsection}
\renewcommand{\subsection}{
    \stepcounter{mysubsection}
    \oldsubsection
}
\renewcommand{\thesubsection}{\arabic{mysubsection}.}

\definecolor{lightgrey}{rgb}{0.96,0.96,0.96}
\newcommand{\code}[1]{\texttt{\color{black}{#1}}}
\newcommand{\keyword}[1]{\textbf{\color{black}{#1}}}
\newcommand{\example}[1]{\textbf{\medskip\colorbox{lightgrey}{\color{black}{Beispiel: #1}}}}

\begin{document}
\maketitle

\section{Accustoming Yourself to C++}

\subsection{View C++ as a federation of languages}
There are several aspects of C++: C, Object-Oriented C++, Template C++, STL, and even more. The approaches to write efficient C++ vary from aspect to aspect.

\subsection{Prefer \code{const}s, \code{enum}, \code{inline}s to \code{\#define}s}
Whenever possible avoid \code{\#include}. If it's a global constant just use a \code{const} variables in anonymous namespaces. (if you have pointers don't forget to write const twice.) The advantage of this is that using \code{static const} members of classes have limited accessibilty.\\
There is a point about hacking around a limitation of old compilers not accepting initial values in declaration. When you need that you can use an enum which can have a defined value. But don't do this if you work with current compilers.\\
Another point is that you can get rid of macros that replace function calls by using templates at no runtime cost.\\
One point that is missing in the book is that you should use macros as seldomly as possible as IDEs sometimes have a hard time to follow them and an even harder one to refactor them.

\subsection{Use \code{const} whenever possible}
You can easily make pointers and objects immutable using \code{const}. This might allow you to find certain bugs at compile time already. But always be aware of what exactly is const (\code{const T*} vs. \code{T* const}).\\
It is prudent to make return types from operators const as otherwise you could assign to the result of an operation.\\
If you have to write a \code{const} and a non-\code{const} member function you typically do code duplication. If you want to avoid that in the non-\code{const} function call the \code{const} function and cast away the \code{const} on the return value. (Don't forget to cast \code{this} to a const object to get the other implementation.

\subsection{Make sure that objects are initialized before they're used} \label{ssec:objInitialization}
Either read up on when your data is guaranteed to be initialized or always initialize all your data.\\
Don't misunderstand initializing with assignment. This easily happens in class constructors. The problem is not with built in data types butwith user-defined types which will first be default constructed and then assigned to.\\
There is a tip to use private functions to initialize class members if too many constructors exist. But this is not accurate anymore - the current approach for this is to use delegating constructors.\\
Another thing are static objects. To get around the problem of the right initialization order we can put them as static variables into functions where they will be initialied before they are used. Until recently this was not thread safe but most compilers guarantee thread safe initialization now.
\newpage

\section{Constructors, Destructors, and Assignment Operators}

\subsection{Know what functions C++ silently writes and calls} \label{ssec:silentFunctions}
If you don't write the functions on your own C++ will write some for you.
\begin{verbatim}
class Empty{};
\end{verbatim}
becomes
\begin{verbatim}
class Empty {
  public:
    Empty() {...} // default constructor
    Empty(const Empty& rhs) {...}
                    // copy constructor
    Empty(Empty&& rhs) {...}
                    // move constructor
    ~Empty() {...} // destructor
    Empty& operator=(const Empty& rhs)
              {...} // copy assignment
    Empty& operator=(Empty&& rhs)
              {...} // move assignment
}
\end{verbatim}
This is augmented from the book's version which covers C++ from pre C++11 / move semantics.

The created constructors and assignment operators just copy/assign/move all the members from the other object.

But the compiler will only create these functions for you if the created code is legal and it has any chance of making sense. This means that if you use a reference or \code{const} member these functions will not be created.

\subsection{Explicitly disallow the use of compiler-generated function you do not want} \label{ssec:DeleteCompilerFunctions}
Some objects should not be copyable and you'd like the error to arise at compile time already.\\
The book proposes to make the copy constructor and copy assignment operator \code{private} and not implement them. This prohibits all other code to use them and if someone were to use them inside the class compilation would fail at link time.\\
You could also have a base class that defines these functions private and inherit from it to make this more explicit.

Not covered by the book is that modern C++ has a cleaner approach to this: you can delete unwanted functions:
\begin{verbatim}
	class NoCopy {
	  public:
	    NoCopy() {...}
	    NoCopy(const NoCopy&) = delete;
	    NoCopy& operator=(const NoCopy&)
	                          = delete;
	}
\end{verbatim}
With modern C++ you might think about declaring a move constructor and assinment operator. This doesn't copy your object but moves it around.

\subsection{Declare destructors virtual in polymorphic base classes} \label{ssec:virtDtor}
Assume you have a file system interface that hides all the specialities of each operating system. There is a factory function that returns you a pointer to an object of the interface type with the contract that you delete that pointer after use.\\
If your destructor is non-virtual \code{delete} C++ specifies undefined behavior - typically that only destroys the base class part while the derived part is never destroyed.\\
The easy solution to get around this is to define the destructor \code{virtual}. With this the call to the base class destructor will be dynamically dispatched and routed to the derived class destructor (which in turn calls the base's destructor).

If a class does not define a virtual destructor it is often not meant to extended. Although the reasons for not declaring a \code{virtual} destructor might be different as well:
\begin{itemize}
	\item Not declaring any virtual functions enables you to pass your objects to languages like C or FORTRAN as they don't know \code{virtual}.
	\item Having \code{virtual} functions in an object increases the size of that object by one pointer (typically).
\end{itemize}
This means that in gener you declare a \code{virtual} destructor if your class has at least one \code{virtual} function.

Be aware that standard library classes like \code{string}, \code{vector}, \code{map}, etc. all don't have a \code{virtual} destructor.

Declare a pure \code{virtual} destructor if you are declaring an abstract class that has no abstract functions - that makes it abstract.\\
But be aware that a definition of the destructor is necessary anyway. The compiler will generate a call to the base class's destructor from each derived class's destructor.

\subsection{Prevent exceptions from leaving destructors} \label{ssec:ExceptionsInDtors}
Say you have a database connection class \code{DbConnection}. The conneciton is established in the constructor and closed in the destructor.\\
What happens if the connection cannot be closed - e.g. because the connection is lost? You throw an exception which the calling code needs to handle?\\
For normal code flow this seems fine and testing will make it seem ok. But if your connection's destructor is called because another exception unwinds the stack throwing another excpetion is trouble - C++ either defines program termination or undefined behavior.\\

There are two primary ways to resolve this issue:
\paragraph{Terminate the program}
If you are at a point where you want to throw an exception in the destructor and you cannot continue normal execution then (log the error and) terminate the program the proper way (\code{std::abort();}).
\paragraph{Swallow the exception}
Sometimes it's better though to (log and) swallow an exception than to have/run the risk of premature program termination.
\paragraph{}
An alternative approach would be to extract the functionality in the destructor to a funciton - e.g. \code{close} that may throw. That way the client can handle the exception if he wants to. And if \code{clode} wasn't called on object destruction the destructor will call it and use one of the above approaches to handle the exception.

\subsection{Never call virtual functions during construction or destruction}
In constructing a derived class first the base class constructor is called. But during the base class constructor all \code{virtual} functions behave as though it were a base class object. The virtual function therefore will not dispatch to the class we want to create but only to the class in whose constructor we are.

It doesn't only behave this way - the type of the object during construction is always of the class whose constructor is currently running. And the same applies during destruction.

An alternatives to \code{virtual} functions is to pass all arguments to the base constructor so he doesn't need to pass the control flow to the derived class.\\
Conversely if that is not possible because the base class doesn't have all information the functionality belongs in the derived class constructor.

\subsection{Have assignment operators return a reference to \code{*this}}
To enable assignment chains like \code{x = y = z = 14} we need to return a reference to \code{*this} from the assignment operator. This rule applies to all assignment operators, not just the standard ones. E.g. \code{+=}, \code{-=}, \code{*=}, \code{/=},...

\subsection{Handle assignment to self in \code{operator=}}
Always think about an assignment to self (\code{x = x}) when writing an assignment operator. It might happen easier than you think that an object is assigned to itself (e.g. via pointers).

It is particularly important if the assinment operator releases resouces of the object previously held (like freeing memory). If the order is wrong you might free the memory and afterwards copy the already freed memory.\\
This situation typically means that your assignment operator isn't exception safe either. That means if an exception leaves the assinment operator nothing has changed in the object.\\
If you manage to make assinment operators exception save it typically means that they're also self-assignment save.

An alternative is to use the "copy and swap" technique which also relates to exception safe code (Item \ref{ssec:ExceptionSafeCode}). The idea is to use the copy constructor to create a new instance and then use a swap function that typically just swaps all members of the objects.

If you're an efficiency buff and think that assinment to self occurrs often enough just use the following:
\begin{verbatim}
	MyClass& operator=(const MyClass& rhs) {
	  if (this == &rhs)
	    return *this;
	  ...
	}
\end{verbatim}

\subsection{Copy all parts of an object}
Be aware that if you define your own copy- (and move-) constructor you need to make sure that all members are copied. The compiler won't do that for you anymore. Whenever you add a member you need to add it to all the constructors.

The same goes for derived classes. You need to actively call the copy constructor of the base class.

If you want to combine copy assinment and copy construction do it in a third common funciton and not by calling each other.
\newpage

\section{Resource Management}

\subsection{Use objects to manage resources}
Put all resources you manage inside an object that makes sure the resource is released again. An example for this are smart pointers. The concept is named RAII (Resource Acquisition is Initialization).

There are two critical aspects:
\begin{itemize}
	\item Turn acquired resources over to resource management objects immediately.
	\item Resource management objects use the destructor to release the resource. (See item \ref{ssec:ExceptionsInDtors}.)
\end{itemize}

The book now talks about details of \code{std::auto\_ptr} which are outdated as of C++11 anyways: use \code{std::shared\_ptr} and \code{std::unique\_ptr}. It also goes on about these concepts not working for dynamic arrays with is outdated as well.

Be aware that the \code{std::shared\_ptr} cannot break cycles of pointers and delete those resources. These will stay valid indefinitely.

\subsection{Think carefully about copying behavior in resource-managing classes}
Typically resources are non-copiable and therefore resource-managing classes are in a dilemma of what to do when being copied:
\begin{itemize}
	\item Prohibit copying by deleting the copying functions (see item \ref{ssec:DeleteCompilerFunctions}).
	\item Release the resource when the reference-count to the managing resource is zero. (This is what \code{std::shared\_ptr} does.)
	\item If the resource is copyable the managing class might be copied as well.
	\item In rare cases you transfer the ownership of the resource. This is what the outdated \code{std::auto\_ptr} did.
\end{itemize}

\subsection{Provide access to raw resources in resource-managing classes} \label{ssec:rawResources}
As some API function might rely on the raw resource rather than the resource managing classes you need access to the raw resource (e.g. define a \code{get()} function).

If the resource has member functions of its own it might be reasonable to define \code{operator->}.

Finally if you want implicit conversion of the managing type to the resource type you want to define \code{operator Container() const}. But this might also lead to unwanted conversions and with that duplication of the resource.

\subsection{Use the same form in corresponging uses of \code{new} and \code{delete}}
If you use \code{new} to allocate an array you need to use \code{delete []}. The reason is that \code{new} default constructs these objects and \code{delete []} destructs all objects. If you only use \code{delete} you will typically only deconstruct the first element.

\subsection{Store \code{new}ed objects in smart pointers in standalone statements}
Always store \code{new}ed objects and arrays in smart pointers - either \code{unique\_ptr} or \code{shared\_ptr}.

Also make sure allocation and passing into the smart pointer is the only thing you do on that line. Otherwise you might have a leak in between. Look at:

\code{processElement(std::shared\_ptr<Element>(\\new Element()), getPosition());}

If \code{new Element()} is executed first and \code{getPosition()} second it's possible that it throws an exception before the \code{new}ed \code{Element} is passed to the smart pointer.

Better even: Use \code{std::make\_unique} resp. \code{std::make\_shared}.
\newpage

\section{Designs and Declarations}

\subsection{Make interface easy to use correctly and hard to use incorrectly}
Making a date API is hard. How do you create a date object. Is it \code{Date(day, month, year)} or rather \code{Date(month, day, year)}?

A solution is to have structs for each \code{Day}, \code{Month} and \code{Year} and build the Date from those. This would then look like this:

\code{Date d(Day(30), Month(10), Year(1982));}

If the order is wrong your compile will complain and you will know instead of having a bug.\\
You can go further and restrict the number of months by defining the constructor private and public static functions to create each one. See item \ref{ssec:objInitialization} if you want to create static objects instead of functions.

Try to make your API behave consisten to built-in types and classes in the STL. Making things \code{const} also limits what (wrong) clients of an API can do with objects.

It is typically a good idea to return smart pointers if you need some destruction work done reliably. You can also have your own destructor on those instead of using the default \code{delete}.

\subsection{Treat \code{class} design as type design}
If you design your class rather than let it happen you end up with better code. Questions you should ask yourself while writing include:
\begin{itemize}
	\item What does the lifetime of your object look like?\\
		Is it created by static functions \code{from...(...)} (possibly returning smart pointers?), constructor(s)? Does the client need to call \code{close}? Also see chapter \ref{sec:newDelete}
	\item What's the difference between construction and assignment? See item \ref{ssec:silentFunctions}.
	\item What does it mean for objects of your new type to be passed by value?\\
		What all is copied in the copy constructor?
	\item What are the restrictions on legal values for your new type?\\
		What are the invariants? What happens when they are (about to be) violeted? Do you want to enumerate the states as static functions?
	\item Does your new type fit into an inheritance graph?
		What are designs and restrictions (e.g. non-virtual functions) of classes you want to inherit from? What functions are virtual in your class so it can be extended? See item \ref{ssec:virtDtor}
	\item What kind of type conversions are allowed for your new type?\\
		Are there types that you want to convert to implicitly or explicitly? See item \ref{ssec:rawResources}
	\item What operators and functions make sense for the new type?\\
		What functions should you provide? Which are members - which are free? See items \ref{ssec:freeFunctions}, \ref{ssec:freeFunctions2}, \ref{ssec:freeTemplateFunctions}.
	\item What standard functions should be disallowed? See item \ref{ssec:DeleteCompilerFunctions}.
	\item Who should have access to the members of your new type?\\
		Which are \code{public}, \code{protected} and \code{private}? Which should have getters - which setters? Prefer getters and setters to direct access for encapsulation reasons.
	\item What is the "undeclared interface" of your new type?\\
		What performance and invariants do you guarantee? Are you thread safe?
	\item How general is your new type?\\
		Should you define a supertype with subtypes? Should it be a template?
	\item Is a new type really what you need?\\
		... or would some functions do the job just as well?
\end{itemize}

\subsection{Prefer pass-by-reference-to-\code{const} to pass-by-value}
Copy by value can incurr a lot of overhead on copy constructions. Furthermore there is danger of slicing your object if a super type is passed by value - your value is cut and only the super type's parts survive.

At least back then it was reasonable to also pass small user defined types by reference as some compilers didn't want to put them in registers whatever you do.

But the bottom line is: Unless you have a good reason pass everything by reference except for built-in types and STL iterators.

\subsection{Don't try to return a reference when you must return an objects}
Never return a reference to const that was created inside the function. The value will seize to exist once the function returns. Also resist the temptation to return references to statics in your operators. If you try to do \code{a * b == c * d} and your \code{operator*} returns a reference to static this will always be true as the reference is twice the same.

Be aware that there is return value optimization (RVO) in C++. This means that when possible the return value will be not be copied back to where you called the function but it will be constructed there.

\subsection{Declare data members \code{private}}
Typically you want to hide your implementation behind interfaces (of functions) so you can easily replace implementations without changing client code - this is called Encapsulation.

As \code{protected} members can be used in any sub-classes and they are no better than \code{public} members. This means that you want to use \code{private} for all class members.

\subsection{Prefer non-member non-friend functions to member functions} \label{ssec:freeFunctions}
It's a wrong assumption that your functions and your data need to go as close together as possible. The data should be as encapsulated as possible. This means that we want as few functions to change as possible when the implementation changes.

So that means that all functions that don't need access to the private members of a class should be free functions which means non-member non-friend functions. It has proven to be reasonable to put them in the same namespace as their classes. A client can even contribute his own functions to that namespace.

These functions are all utility functions. This means that the client could achieve all the functionality without that function but it would be more complex.

Another advantage of free functions is that you can split up large collections of them across multiple header files so you only have to be compile dependent on the ones you need.

\subsection{Declare non-member functions when type converstions should apply to all parameters} \label{ssec:freeFunctions2}
If you declared member functions you only get type conversion on the arguments but not on the object you call them on. This seems obvious in most circumstances - but with \code{operator}s it's a different story.

A multiplication that is implemented as a member function will only work if the object on the left of the operator doesn't need conversion. To solve this problem it's best to implement non-member \code{operator}s because they can have type conversion on both arguments.

\subsection{Consider support for a non-throwing \code{swap}}
The default \code{std::swap} is inefficient as it has one copy construction,  one destruction and two copy assignments. We therefore often want to supply an efficient \code{swap} operation. Make sure you follow these rules:
\begin{itemize}
	\item Make sure \code{swap} doesn't throw. It makes implementing the strong exception guarantee really easy. (See item \ref{ssec:ExceptionSafeCode}).
	\item Write a public member \code{swap} that swaps it with another object.
	\item Also write a non-member \code{swap} in the same namespace as the object.
	\item If you write a class (not a template) specialize \code{std::swap} for your object.
\end{itemize}
Calls to \code{swap} should never be qualified (e.g. with \code{std::}) and have \code{using std::swap;} before the call. The compiler will then figure out the correct \code{swap} function to call using ADL and function selection mechanics.
\newpage

\section{Implementations}

\subsection{Postpone variable definitions as long as possible}

\subsection{Minimize casting}

\subsection{Avoid returning "handles" to object internals}

\subsection{Strive for exception-safe code}
\label{ssec:ExceptionSafeCode}

\subsection{Understand the ins and outs of inlining}

\subsection{Minimize compilation dependencies between files}
\newpage

\section{Inheritance and Object-Oriented Design}

\subsection{Make sure public inheritance models "is-a".}

\subsection{Avoid hiding inherited names}

\subsection{Differentiate between inheritance of interface and inheritance of implementation}

\subsection{Consider alternatives to virtual functions}

\subsection{Never redefine an inherited non-virtual function}

\subsection{Never redefine a function's inherited default parameter value}

\subsection{Model "has-a" or "is-implemented-in-terms-of" through composition}

\subsection{Use private inheritance judiciously}

\subsection{Use multiple inheritance judiciously}
\newpage

\section{Templates and Generic Programming}

\subsection{Understand implicit interfaces and compile-time polymorphism}

\subsection{Understand the two meanings of \code{typename}}

\subsection{Know how to access names in templatized base classes}

\subsection{Factor parameter-independent code out of templates}

\subsection{Use member function templates to accept "all compatible types"}

\subsection{Define non-member functions inside templates when type conversions are desired} \label{ssec:freeTemplateFunctions}

\subsection{Use traits classes for information about types}

\subsection{Be aware of template metaprogramming}
\newpage

\section{Customing \code{new} and \code{delete}} \label{sec:newDelete}

\subsection{Understand the behavior of the \code{new}-handler}

\subsection{Understand when it makes sense to replace \code{new} and \code{delete}}

\subsection{Adhere to convention when writing \code{new} and \code{delete}}

\subsection{Write placement \code{delete} if you write placement \code{new}}
\newpage

\section{Miscellany}

\subsection{Pay attention to compiler warnings}

\subsection{Familiarize yourself with the standard library, including \code{TR1}}

\subsection{Familiarize yourself with Boost}
\newpage

\section{Beyond Effective C++}


\end{document}
