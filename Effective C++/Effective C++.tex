\documentclass[a4paper, twocolumn]{article}

\usepackage{amsmath, amsthm, amssymb, amsfonts} 
\usepackage{color}
\usepackage[utf8]{inputenc}
\usepackage{multirow}
\usepackage{graphics}
\usepackage[pdftex]{hyperref}
\usepackage{moreverb}

\usepackage[novbox]{pdfsync}

\title{Summary: Effective C++ \\
	\large
	55 Specific Ways to Improve Your Programs and Designs
	Scott Meyers}
\author{Stefan Götschi}

% Commands

\let\oldsubsection\subsection

\newcounter{mysubsection}
\renewcommand{\subsection}{
    \stepcounter{mysubsection}
    \oldsubsection
}
\renewcommand{\thesubsection}{\arabic{mysubsection}.}

\definecolor{lightgrey}{rgb}{0.96,0.96,0.96}
\newcommand{\code}[1]{\texttt{\color{black}{#1}}}
\newcommand{\keyword}[1]{\textbf{\color{black}{#1}}}
\newcommand{\example}[1]{\textbf{\medskip\colorbox{lightgrey}{\color{black}{Beispiel: #1}}}}

\begin{document}
\maketitle

\section{Accustoming Yourself to C++}

\subsection{View C++ as a federation of languages}
There are several aspects of C++: C, Object-Oriented C++, Template C++, STL, and even more. The approaches to write efficient C++ vary from aspect to aspect.

\subsection{Prefer \code{const}s, \code{enum}, \code{inline}s to \code{\#define}s}
Whenever possible avoid \code{\#include}. If it's a global constant just use a \code{const} variables in anonymous namespaces. (if you have pointers don't forget to write const twice.) The advantage of this is that using \code{static const} members of classes have limited accessibilty.\\
There is a point about hacking around a limitation of old compilers not accepting initial values in declaration. When you need that you can use an enum which can have a defined value. But don't do this if you work with current compilers.\\
Another point is that you can get rid of macros that replace function calls by using templates at no runtime cost.\\
One point that is missing in the book is that you should use macros as seldomly as possible as IDEs sometimes have a hard time to follow them and an even harder one to refactor them.

\subsection{Use \code{const} whenever possible}
You can easily make pointers and objects immutable using \code{const}. This might allow you to find certain bugs at compile time already. But always be aware of what exactly is const (\code{const T*} vs. \code{T* const}).\\
It is prudent to make return types from operators const as otherwise you could assign to the result of an operation.\\
If you have to write a \code{const} and a non-\code{const} member function you typically do code duplication. If you want to avoid that in the non-\code{const} function call the \code{const} function and cast away the \code{const} on the return value. (Don't forget to cast \code{this} to a const object to get the other implementation.

\subsection{Make sure that objects are initialized before they're used}
Either read up on when your data is guaranteed to be initialized or always initialize all your data.\\
Don't misunderstand initializing with assignment. This easily happens in class constructors. The problem is not with built in data types butwith user-defined types which will first be default constructed and then assigned to.\\
There is a tip to use private functions to initialize class members if too many constructors exist. But this is not accurate anymore - the current approach for this is to use delegating constructors.\\
Another thing are static objects. To get around the problem of the right initialization order we can put them as static variables into functions where they will be initialied before they are used. Until recently this was not thread safe but most compilers guarantee thread safe initialization now.
\newpage

\section{Constructors, Destructors, and Assignment Operators}

\subsection{Know what functions C++ silently writes and calls}

\subsection{Explicitly disallow the use of compiler-generated function you do not want}

\subsection{Declare destructors virtual in polymorphic base classes}

\subsection{Prevent exceptions from leaving destructors}

\subsection{Never call virtual functions during construction or destruction}

\subsection{Have assignment operators return a reference to \code{*this}}

\subsection{Handle assignment to self in \code{operator=}}

\subsection{Copy all parts of an object}
\newpage

\section{Resource Management}

\subsection{Use objects to manage resources}

\subsection{Think carefully about copying behavior in resource-managing classes}

\subsection{Provide access to raw resources in resource-managing classes}

\subsection{Use the same form in corresponging uses of \code{new} and \code{delete}}

\subsection{Store \code{new}ed objects in smart pointers in standalone statements}
\newpage

\section{Designs and Declarations}

\subsection{Make interface easy to use correctly and hard to use incorrectly}

\subsection{Treat \code{class} design as type design}

\subsection{Prefer pass-by-reference-to-\code{const} to pass-by-value}

\subsection{Don't try to return a reference when you must return an objects}

\subsection{Declare data members \code{private}}

\subsection{Prefer non-member non-friend functions to member functions}

\subsection{Declare non-member functions when type converstions should apply to all parameters}

\subsection{Consider support for a non-throwing \code{swap}}
\newpage

\section{Implementations}

\subsection{Postpone variable definitions as long as possible}

\subsection{Minimize casting}

\subsection{Avoid returning "handles" to object internals}

\subsection{Strive for exception-safe code}

\subsection{Understand the ins and outs of inlining}

\subsection{Minimize compilation dependencies between files}
\newpage

\section{Inheritance and Object-Oriented Design}

\subsection{Make sure public inheritance models "is-a".}

\subsection{Avoid hiding inherited names}

\subsection{Differentiate between inheritance of interface and inheritance of implementation}

\subsection{Consider alternatives to virtual functions}

\subsection{Never redefine an inherited non-virtual function}

\subsection{Never redefine a function's inherited default parameter value}

\subsection{Model "has-a" or "is-implemented-in-terms-of" through composition}

\subsection{Use private inheritance judiciously}

\subsection{Use multiple inheritance judiciously}
\newpage

\section{Templates and Generic Programming}

\subsection{Understand implicit interfaces and compile-time polymorphism}

\subsection{Understand the two meanings of \code{typename}}

\subsection{Know how to access names in templatized base classes}

\subsection{Factor parameter-independent code out of templates}

\subsection{Use member function templates to accept "all compatible types"}

\subsection{Define non-member functions inside templates when type conversions are desired}

\subsection{Use traits classes for information about types}

\subsection{Be aware of template metaprogramming}
\newpage

\section{Customing \code{new} and \code{delete}}

\subsection{Understand the behavior of the \code{new}-handler}

\subsection{Understand when it makes sense to replace \code{new} and \code{delete}}

\subsection{Adhere to convention when writing \code{new} and \code{delete}}

\subsection{Write placement \code{delete} if you write placement \code{new}}
\newpage

\section{Miscellany}

\subsection{Pay attention to compiler warnings}

\subsection{Familiarize yourself with the standard library, including \code{TR1}}

\subsection{Familiarize yourself with Boost}
\newpage

\section{Beyond Effective C++}


\end{document}
