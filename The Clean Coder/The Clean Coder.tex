\documentclass[a4paper, twocolumn]{article}

\usepackage{amsmath, amsthm, amssymb, amsfonts} 
\usepackage{color}
%\usepackage{german}
\usepackage[utf8]{inputenc}
%\usepackage{multicol}  
\usepackage{multirow}
%\usepackage{dsfont} 
%\usepackage[rflt]{floatflt}  
\usepackage{graphics}
\usepackage{epsfig} 
\usepackage[pdftex]{hyperref}
\usepackage{moreverb}

\usepackage[novbox]{pdfsync}

\title{Summary: The Clean Coder \\
	\large
	A Code of Conduct for Professional Programmers
	Robert C. Martin}
\author{Stefan Götschi}

% Commands

\definecolor{lightgrey}{rgb}{0.96,0.96,0.96}
\newcommand{\code}[1]{\texttt{\color{black}{#1}}}
\newcommand{\keyword}[1]{\textbf{\color{black}{#1}}}
\newcommand{\example}[1]{\textbf{\medskip\colorbox{lightgrey}{\color{black}{Beispiel: #1}}}}

\begin{document}
\maketitle

\section{Professionalism}
Professionalism means taking responsibility that would otherwise lies with the employer.
Taking responsibilty typically means that you only ship - better even only check in - code that you are reasonably sure it works. This means you ran all the tests on it and it passed them all.

\section{Saying No}

\section{Saying Yes}

\section{Coding}

\section{Test Driven Development}

\section{Practicing}

\section{Acceptance Testing}

\section{Testing Strategies}

\section{Time Management}

\section{Estimation}

\section{Pressure}

\section{Collaboration}

\section{Teams and Projects}

\section{Mentoring, Apprenticeship and Craftsmanship}

\end{document}
